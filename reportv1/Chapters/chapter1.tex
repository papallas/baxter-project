\chapter{Introduction}
\label{chapter1}
\setcounter{page}{1}
\section{Aim}
To produce a piece of software using the robotic operating system (ROS) that will allow Baxter to interact with people and give them sweets within a sweet shop setting. The personal aim of this project is to learn the algorithms and software systems used in autonomous systems whilst doing this project. Practically, the aim is to allow the School of Computing to be able to use this system as an interactive demonstration on Open Days.
\section{Objectives}
\begin{enumerate}
    \item{Develop a vision system to allow Baxter to recognise sweets and the container they are in}
    	\begin{enumerate}
		\item{Deliverables - Software to allow Baxter to be able to identify and locate a bowl and individual sweet shapes/colours (depending on the recognition method). A section on the report comparing vision methods and deciding the most appropriate one to use.} 
		\item{Provide a section in the results section of the report showing the successful recognition of the objects by Baxter visualized using rViz/images.}
	\end{enumerate}	
    \item{Develop an algorithm for the singulation of sweets}
    	\begin{enumerate}
		\item{Deliverables - Code to allow Baxter to singulate specified sweets from a bowl. A section on the report comparing sweet singulation algorithms, determining which one was the most efficient to use.}
		\item{Provide results from the sweet singulation algorithm running in different environments and scenarios.}
	\end{enumerate}	
    \item{Develop a method for Baxter to manipulate and count out change}
    	\begin{enumerate}
		\item{Deliverables - code allowing Baxter to identify and recognise certain coins/notes and then being able to manipulate the money to grab a particular amount.}
		\item{Provide results in the report from the money manipulation algorithm running in 		various environments.}
	\end{enumerate}	
    \item{Develop a method of Human-Interaction}
    	\begin{enumerate}
		\item{Deliverables - code allowing Baxter to interact with a human customer and 		recognise some way of telling which sweets the human wants.}
		\item{Provide evidence of different methods of robot-human interaction tested in the 		report and an analysis on which method works best.}
	\end{enumerate}	
\end{enumerate}
\section{Methodology}
Deliverables to be included:
\begin{itemize}
	\item{A Github repository of working code to allow Baxter to serve sweets to a customer with instructions on how to set up. The Github should be separated into different software packages, for example: perception, manipulation and interaction.}
	\item{A working demo of the Robot Shopkeeper to be used at Open Days in the School of Computing, with a video of the working demo.}
	\item{An analysis of object recognition methods - using OpenCV or point cloud methods.}
	\item{An analysis of object manipulation methods for Baxter to grab and separate sweets into the specified type and amount.}
	\item{An analysis of human interaction methods, with a choice on which is best/appropriate to implement.}
	\item{For each analysis above, include referencing and comparisons to academic papers, that will be studied at the appropriate points in the project.}
	\item{Aside from analyses, results from the techniques used should be provided in an appropriate form, such as images/graphs and video recordings.}
\end{itemize}
\section{Tasks, milestones and timeline}
Here are each of the tasks explained in more detail:
\newline
\newline
\textbf{Project setup} - Setup the ROS system and look learn the basics of the software. Complete ROS and Baxter tutorials and make some example programs to test and move Baxter in different ways.
\newline
\newline
\textbf{Setup Kinect recognition} - Setup the Kinect with Baxter and look into different computer vision methods to recognise the bowl and sweets and their positions in 3d space.
\newline
\newline
\textbf{Sweet singulation} - Trial multiple object manipulation algorithms to allow Baxter to grab and select specified sweet combinations.
\newline
\newline
\textbf{Human interaction} - Create some form of human interaction between Baxter and a human to allow the human to ask for specific sweets. This could be gestures or voice control based.
\newline
\newline
\textbf{Money Manipulation} - Develop some vision system to recognise paper money or change and then work on algorithms for Baxter to handle/count out money to give to a person.
\newline
\newline
\textbf{Extra Features} - Allow time to further improve upon previous steps or pick one dedicated topic to further go into and develop.
\newline
\newline
\textbf{Report Completion} - Allow the last three weeks to further improve upon the project report.
\newline
\newline
\textbf{Background Research} - Throughout each stage of the project, make sure at least two academic papers have been researched, with summary paragraphs on each for use in the report.
\newline
\newline
\textbf{Project Report} - At each week, take a look at project report deadlines that need to be met. Make sure parts of the report are written at the time they are completed.
\newline
\newline
Here is a Gantt chart representing the order of the tasks for the project:
\newline
\definecolor{barblue}{RGB}{153,204,254}
\definecolor{groupblue}{RGB}{51,102,254}
\definecolor{linkred}{RGB}{165,0,33}
\begin{ganttchart}[
    canvas/.append style={fill=none, draw=black!5, line width=.75pt},
    hgrid style/.style={draw=black!5, line width=.75pt},
    vgrid={*1{draw=black!5, line width=.75pt}},
    title/.style={draw=none, fill=none},
    title label font=\bfseries\footnotesize,
    title label node/.append style={below=7pt},
    x unit =0.7cm,
    include title in canvas=false,
    bar label font=\mdseries\small\color{black!70},
    bar label node/.append style={left=2cm},
    bar/.append style={draw=none, fill=black!63},
    bar incomplete/.append style={fill=barblue},
    bar progress label font=\mdseries\footnotesize\color{black!70},
    group incomplete/.append style={fill=groupblue},
    group left shift=0,
    group right shift=0,
    group height=.5,
    group peaks tip position=0,
    group label node/.append style={left=.6cm},
    group progress label font=\bfseries\small,
    link/.style={-latex, line width=1.5pt, linkred},
    link label font=\scriptsize\bfseries,
    link label node/.append style={below left=-2pt and 0pt}
  ]{1}{12}
  \gantttitle[
    title label node/.append style={below left=7pt and -3pt}
  ]{WEEKS:\quad1}{1}
  \gantttitlelist{2,...,12}{1} \\
  \ganttgroup[]{Version 0: Basic Solution}{1}{5} \\
  \ganttbar[
    name=WBS1A
  ]{\textbf{Project Setup}}{1}{1} \\
  \ganttbar[
    name=WBS1B
  ]{\textbf{Setup Kinect Recognition}}{2}{4} \\
  \ganttbar[
    name=WBS1C
  ]{\textbf{Sweet Singulation}}{4}{5} \\[grid]
  \ganttgroup{Version 1: Main Solution}{6}{9} \\
  \ganttbar{\textbf{Human Interaction}}{6}{7} \\
  \ganttbar{\textbf{Money Manipulation}}{8}{9} \\
  \ganttgroup{Version 2: Project Completion}{10}{12} \\
  \ganttbar{\textbf{Extra Features}}{10}{12} \\
  \ganttbar{\textbf{Report Completion}}{10}{12} \\[grid]
  \ganttgroup{Background Research}{1}{12} \\
  \ganttgroup{Project Report}{1}{12} \\
\end{ganttchart}